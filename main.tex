\documentclass{beamer}

\usetheme{CambridgeUS}
\setbeamertemplate{blocks}[rounded][shadow=true]
\defbeamertemplate{itemize item}{ball}{\raise0.2pt\beamer@usesphere{item projected}{bigsphere}}
\defbeamertemplate{itemize subitem}{ball}{\raise0.2pt\beamer@usesphere{subitem projected}{smallsphere}}
\defbeamertemplate{itemize subsubitem}{ball}{\raise0.2pt\beamer@usesphere{subsubitem projected}{smallsphere}}
\setbeamercolor{item projected}{bg=red}
\setbeamercolor{subitem projected}{bg=red}
\setbeamercolor{title}{bg=red!65!black, fg=white}

\usepackage[euler]{textgreek}
\usepackage{amsmath}
\usepackage{pdfpages}
\usepackage{hyperref}
\usepackage{tikz}
\usepackage{natbib}
\usepackage{graphicx}
\usepackage{ragged2e}
\usepackage{tcolorbox}
\usepackage{caption}
\usepackage{color}
\renewcommand{\figurename}{\color{red}{Figure}}
\usepackage{ragged2e}
\usepackage{array}
\usepackage{booktabs}

\newtheorem{prop}{Proposition}[section]

\makeatletter
\renewcommand{\itemize}[1][]{%
  \beamer@ifempty{#1}{}{\def\beamer@defaultospec{#1}}%
  \ifnum \@itemdepth >2\relax\@toodeep\else
    \advance\@itemdepth\@ne
    \beamer@computepref\@itemdepth% sets \beameritemnestingprefix
    \usebeamerfont{itemize/enumerate \beameritemnestingprefix body}%
    \usebeamercolor[fg]{itemize/enumerate \beameritemnestingprefix body}%
    \usebeamertemplate{itemize/enumerate \beameritemnestingprefix body begin}%
    \list
      {\usebeamertemplate{itemize \beameritemnestingprefix item}}
      {\def\makelabel##1{%
          {%
            \hss\llap{{%
                \usebeamerfont*{itemize \beameritemnestingprefix item}%
                \usebeamercolor[fg]{itemize \beameritemnestingprefix item}##1}}%
          }%
        }%
      }
  \fi%
  \beamer@cramped%
  \justifying% NEW
  %\raggedright% ORIGINAL
  \beamer@firstlineitemizeunskip%
}
\makeatother

\newcounter{saveenumi}
\newcommand{\seti}{\setcounter{saveenumi}{\value{enumi}}}
\newcommand{\conti}{\setcounter{enumi}{\value{saveenumi}}}

\resetcounteronoverlays{saveenumi}

\renewcommand{\baselinestretch}{1.35}
\renewcommand{\arraystretch}{0.7}



\title[Speculative Bubbles]{Insert Title Here}
\author[K.Vasilopoulos ]{Kostas Vasilopoulos}
\institute[]{\scriptsize Department of Economics, Lancaster University Management School
}

%\date{AMEF April 21, 2017}

\AtBeginDocument{\setlength{\abovedisplayskip}{0pt}}
\addtobeamertemplate{block begin}{\setlength{\abovedisplayskip}{0pt}}

\usebackgroundtemplate{

\tikz[overlay,remember picture] 
  \node[opacity=0.15, at=(current page.center east),anchor=center east,inner sep=0pt]  {\includegraphics[width=120,height=140]{lancaster_back2}};}
  
 %%%%%%%%%%%%%%%%%%%%%%%%%%%%%%%%%%%%%%%%%%%%%%%%%%%%%%%%%%%%%%%%%%
%%%%%%%%%%%%%%%%%%%%%%%%%%%%%%%%%%%%%%%%%%%%%%%%%%%%%%%%%%%%%%%%%%%%%
%%%%%%%%%%%%%%%%%%%%%%%%%%%%%%%%%%%%%%%%%%%%%%%%%%%%%%%%%%%%%%%%%%%%%%

% Let's get started
\begin{document}
{
\usebackgroundtemplate{}
\begin{frame}
  \titlepage
\end{frame}
}

\begin{frame}{Outline}
  \tableofcontents
  % You might wish to add the option [pausesections]
\end{frame}

% Section and subsections will appear in the presentation overview
% and table of contents.
\section{First Main Section}

\subsection{First Subsection}

\begin{frame}{First Slide Title}{Optional Subtitle}
  \begin{itemize}
  \item {
    My first point.
  }
  \item {
    My second point.
  }
  \end{itemize}
\end{frame}

\subsection{Second Subsection}

{
\usebackgroundtemplate{}
\begin{frame}[plain,t]
    Text
\end{frame}
}
% You can reveal the parts of a slide one at a time
% with the \pause command:
\begin{frame}{Second Slide Title}
  \begin{itemize}
  \item {
    First item.
    \pause % The slide will pause after showing the first item
  }
  \item {   
    Second item.
  }
  % You can also specify when the content should appear
  % by using <n->:
  \item<3-> {
    Third item.
  }
  \item<4-> {
    Fourth item.
  }
  % or you can use the \uncover command to reveal general
  % content (not just \items):
  \item<5-> {
    Fifth item. \uncover<6->{Extra text in the fifth item.}
  }
  \end{itemize}
\end{frame}

\section{Second Main Section}

\subsection{Another Subsection}

\begin{frame}{Blocks}
\begin{block}{Block Title}
You can also highlight sections of your presentation in a block, with it's own title
\end{block}
\begin{theorem}
There are separate environments for theorems, examples, definitions and proofs.
\end{theorem}
\begin{example}
Here is an example of an example block.
\end{example}
\end{frame}

% Placing a * after \section means it will not show in the
% outline or table of contents.
\section*{Summary}

\begin{frame}{Summary}
  \begin{itemize}
  \item
    The \alert{first main message} of your talk in one or two lines.
  \item
    The \alert{second main message} of your talk in one or two lines.
  \item
    Perhaps a \alert{third message}, but not more than that.
  \end{itemize}
  
  \begin{itemize}
  \item
    Outlook
    \begin{itemize}
    \item
      Something you haven't solved.
    \item
      Something else you haven't solved.
    \end{itemize}
  \end{itemize}
\end{frame}

\bibliographystyle{apalike}
\bibliography{references.bib}

% All of the following is optional and typically not needed. 
\appendix
\section<presentation>*{\appendixname}
\subsection<presentation>*{For Further Reading}

\begin{frame}[allowframebreaks]
  \frametitle<presentation>{For Further Reading}
    
  \begin{thebibliography}{10}
    
  \beamertemplatebookbibitems
  % Start with overview books.

  \bibitem{Author1990}
    A.~Author.
    \newblock {\em Handbook of Everything}.
    \newblock Some Press, 1990.
 
    
  \beamertemplatearticlebibitems
  % Followed by interesting articles. Keep the list short. 

  \bibitem{Someone2000}
    S.~Someone.
    \newblock On this and that.
    \newblock {\em Journal of This and That}, 2(1):50--100,
    2000.
  \end{thebibliography}
\end{frame}

\end{document}


